%Feladatkiiras
\begin{flushleft}
\textsc{\bfseries Miskolci Egyetem}\\
Gépészmérnöki és Informatikai Kar\\
Alkalmazott Matematikai Intézeti Tanszék\hspace*{4cm}\hfil \textbf{Szám:}
\end{flushleft}
\vskip 0.5cm
\begin{center}
\large\textsc{\bfseries Szakdolgozat Feladat}
\end{center}
\vskip 0.5cm
Zajáros Tamás (I0VATO) mérnökinformatikus jelölt részére.\newline

\noindent\textbf{A szakdolgozat tárgyköre:} Frontend fejlesztés, JavaScript keretrendszerek \newline

\noindent\textbf{A szakdolgozat címe:} JavaScript alapú frontend technológiák összehasonlítása\newline

\noindent\textbf{A feladat részletezése:}

\bigskip

A keretrendszerek struktúrájának, mechanizmusainak elemzése. Az azonos problémákra adott megoldások komplexitásának vizsgálata.

\bigskip

Az elérhető JavaScript keretrendszerek (például \textit{ReactJS}, \textit{AngularJS}, \textit{Vue.js}) összehasonlítása. Példaprogramok írása, amelyeken szignifikáns különbség mutatkozik a keretrendszerek megoldásai között.

\vfill

\noindent\textbf{Témavezetõ:} Piller Imre (egyetemi tanársegéd) \newline

% \noindent\textbf{Konzulens(ek):} (akkor kötelezõ, ha a témavezetõ nem valamelyik matematikai tanszékrõl való; de persze lehet egyébként is)\newline

\noindent\textbf{A feladat kiadásának ideje:}\newline

%\noindent\textbf{A feladat beadásának határideje:}

\vskip 2cm

\hbox to \hsize{\hfil{\hbox to 6cm {\dotfill}\hbox to 1cm{}}}

\hbox to \hsize{\hfil\hbox to 3cm {szakfelelõs}\hbox to 2cm{}}

\newpage

\vspace*{1cm}  
\begin{center}
\large\textsc{\bfseries Eredetiségi Nyilatkozat}
\end{center}
\vspace*{2cm}  

Alulírott Zajáros Tamás; Neptun-kód: I0VATO, a Miskolci Egyetem Gépészmérnöki és Informatikai Karának végzõs mérnök informatikus szakos hallgatója ezennel büntetõjogi és fegyelmi felelõsségem tudatában nyilatkozom és aláírásommal igazolom, hogy "JavaScript alapú frontend technológiák összehasonlítása"
címû szakdolgozatom/diplomatervem saját, önálló munkám; az abban hivatkozott szakirodalom
felhasználása a forráskezelés szabályai szerint történt.\\

Tudomásul veszem, hogy szakdolgozat esetén plágiumnak számít:
\begin{itemize}
\item szószerinti idézet közlése idézõjel és hivatkozás megjelölése nélkül;
\item tartalmi idézet hivatkozás megjelölése nélkül;
\item más publikált gondolatainak saját gondolatként való feltüntetése.
\end{itemize}

Alulírott kijelentem, hogy a plágium fogalmát megismertem, és tudomásul veszem, hogy
plágium esetén szakdolgozatom visszautasításra kerül.

\vspace*{3cm}

\noindent Miskolc, 2017. év 11. hó 24. nap

\vspace*{3cm}

\hspace*{8cm}\begin{tabular}{c}
\hbox to 6cm{\dotfill}\\
Hallgató
\end{tabular}



\newpage