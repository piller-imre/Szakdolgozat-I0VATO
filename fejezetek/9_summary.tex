\Chapter{Összegzés}

A dolgozatomban a jelenleg legnépszerűbb és legelterjedtebb JavaScript keretrendszereket hasonlítom össze. A dolgozat első részében az MVC keretrendszerről, a JavaScript és ECMAScript közötti különbségről, a verziókról és történelmi áttekintésről volt szó. Ezután az a JavaScript keretrendszerek néhány tulajdonságát mutattam le, mint előnyök és hátrányok. 
A dolgozat fő témája a keretrendszerek (és könyvtárak) több szempontból való összehasonlítása. Ezek a szempontok a CRUD műveletek (Create, Read, Update, Delete) megvalósítása, template-k és a weboldalak közötti routing működése, szűrők és direktívák használata, form validáció, valamint bejelentkezés és regisztráció. 
A négy webalkalmazás AngularJS, Angular 2, Vue.js, és React.js nyelveken készült el. A back-end részt a MongoDB nevű adatbázis programmal valósítottam meg. A weboldalak az MMA harcosokkal foglalkoznak (MMA Fighters).
Véleményem szerint az AngularJS keretrendszer a legkönnyebben megérthető egy laikus számára is, a tesztelése a különböző részeknek egyszerűen megoldható és a kétirányú adatkötés használata megkönnyíti a fejlesztést, ezenfelül az Angular 2-nél és a React.js-nél a komponensek generálásának lehetősége jelentősen lecsökkentette a fejlesztési időt és sokkal komplexebb alkalmazásokat lehet velük készíteni. A Vue.js pedig egy gyors alternatíva, ami ezeknek a keretrendszereknek a legjobb tulajdonságait ötvözi. A JSX nyelvi elemek és a virtuális DOM miatt a React.js több tanulást igényel.
A form validálás az Angular 2-nél beépített direktívákkal való használata több, mint egyszerű, a külön mezőkhöz és direktívákhoz tartozó hibaüzeneteket is könnyen meg tudja határozni a felhasználó. A Vue.js esetében pedig a vee-validate nevű modullal és a beépített szabályaival könnyíti meg a validációt. Az AngularJS-nél a kód hosszúsága nem a legmegfelelőbb, React.js-nél pedig a kétirányú adatkötés miatt a megoldásomban frissíteni kellett a mező aktuális értékét annak változtatásakor.
Az auth0 szerviz mindegyik keretrendszerhez megtalálható, dokumentációjuk átlátható, követhető. 

További tervek, ötletek \\
A alkalmazások továbbfejlesztése, kibővítése újabb funkciókkal. Tervek között szerepel a képfeltöltés funkciójának megvalósítása, saját Auth0 szervíz bejelentkezési form létrehozása, admin és vendég felület hozzáadása.

