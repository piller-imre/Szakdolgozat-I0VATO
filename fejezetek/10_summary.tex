\Chapter{Összegzés}

A dolgozat első részében az MVC keretrendszerről, a JavaScript és ECMAScript közötti különbségről, a verziókról és történelmi áttekintésről volt szó. Ezután az a JavaScript keretrendszerek előnyeit, hátrányait mutattam be. Ezt követően a mintaalkalmazások specifikációja található meg.
A dolgozat fő témája a keretrendszerek több szempontból való összehasonlítása. Ezek a szempontok a CRUD műveletek megvalósítása, template-k és a weboldalak közötti routing működése, szűrők és direktívák használata, form validáció, valamint bejelentkezés és regisztráció. 
A négy webalkalmazás AngularJS, Angular 2, Vue.js, és React.js keretrendszerekben készült el. A backend részhez a MongoDB nevű programot használtam az adatok tárolására, a Node.js-t a szerver megvalósításához. A weboldalak az MMA harcosokkal foglalkoznak (MMA Fighters).
\\Véleményem szerint az AngularJS keretrendszer a legkönnyebben megérthető egy web-es oldalak világában és technológiáiban nem teljesen jártas ember számára is, a tesztelése a különböző részeknek egyszerűen megoldható és a kétirányú adatkötés használata megkönnyíti a fejlesztést. Ezenfelül az Angular 2-nél és a React.js-nél a komponensek generálásának lehetősége jelentősen lecsökkentette a fejlesztési időt. A React.js használatának megtanulása több időt és erőfeszítést igényel, mindezek mellett jobb megoldás lehet egy nagyobb, komplexebb alkalmazás elkészítéséhez. A Vue.js egy gyors alternatíva, amely ezeknek a keretrendszereknek a legjobb tulajdonságait ötvözi, olyanokat, mint az AngularJS-ben megismert kétirányú adatkötés, JSX támogatás, vagy szerver-oldali renderelés.
Az Auth0 szerviz mindegyik keretrendszerhez megtalálható, dokumentációjuk átlátható, követhető. 

További tervek, ötletek \\
A alkalmazások továbbfejlesztése, kibővítése újabb funkciókkal. Tervek között szerepel a képfeltöltés funkciójának megvalósítása, saját Auth0 szerviz bejelentkezési form létrehozása, admin és vendég felület hozzáadása.

Mindegyik keretrendszernek és technológiának vannak előnyei és hátrányai, amikkel a fejlesztők, és a cégek tisztában vannak, ennek ellenére egy jó lehetőség belelátni ezeknek a technológiáknak a működésébe, használatuknak rejtélyeibe és mindenképpen előny többféle keretrendszer elemeinek és működésének megismerése.

