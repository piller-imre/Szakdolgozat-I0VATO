\Chapter{Keretrendszerek összehasonlítása}

\begin{tabular}{|p{2,3cm}|p{2,5cm}|p{2,95cm}|p{2,8cm}|p{2,65cm}|}
\hline
\textbf{} & \textbf{AngularJS} & \textbf{Angular 2} & \textbf{React.js} & \textbf{Vue.js} \\
\hline
\textbf{Fejlesztő} & Google & Google & Facebook, Instagram & JavaScript könyvtár \\
\hline
\textbf{Github} & commit: 8629, contributor: 1603 & commit: 9009, contributor: 540 & commit: 9425, contributor: 1139 & commit: 2378, contributor: 159 \\
\hline
\textbf{Google találat} & 118 millió  & 103 millió & 180 millió & 23 millió 400 ezer \\
\hline
\textbf{Méret} & 1239 KB, min: 165 KB & 1044 KB, min: 566 KB & 45 KB, min: 6 KB & 272 KB, min: 84 KB \\
\hline
\textbf{Elérhető irodalom} & angularjs.org & angular.io & reactjs.org & vuejs.org \\
\hline
\textbf{Megjelenés Verzió} & 2010.október 1.6.6 & 2014.szeptember 2.0.0 & 2013.március 16.2.0 & 2014.február 2.5.3 \\
\hline
\textbf{Típus} & JavaScript keretrendszer & JavaScript keretrendszer & JavaScript könyvtár & JavaScript keretrendszer \\
\hline
\end{tabular}
\\
\SubSection{Form validáció}
\subsubsection{AngularJS:} "ng-class" és css class-ok együttes használata, direktívák használata a hibaüzenetek megjelenítéséhez (ng-show), egyszerűen megvalósítható. A "number" típusú mezők validálása a "min" és "max" direktívákkal minden további gond nélkül megoldható. Továbbá lehetőség van saját érvényesség ellenörző direktívák létrehozására is. \\
\subsubsection{Angular 2:} Az "*ngIf" és "ngModel" direktívák használatával egyszerűen megvalósítható. Lehetőség van továbbá úgynevezett saját érvényesség ellenőrző direktívák létrehozására, ahol meg lehet határozni, hogy mit ne fogadjon el, esetleg mit fogadjon el a program lehetséges input-ként. A "number" típusú mezők validálása csak külön modullal vagy saját érvényesség ellenőrző direktíva létrehozásával érhető el, amely néha kényelmetlenségeket okozhat. Viszont a különböző mezőknél történő saját hibaüzenet megadásának lehetősége, és az azok közötti automatikus váltás kárpótol emiatt.\\
\subsubsection{Vue.js:} A "vee-validate" modul telepítése szükséges hozzá, elérhető 20 validációs szabály, mint például "alpha\_spaces", amely azt hivatott ellenőrizni, hogy az adott input mező tartalmaz-e szóközt, ha igen, akkor érvényes a mező. Jól használható név mezők esetében, ahol az a fontos, hogy ne lehessen a mezőben numerikus karakter, de lehessen benne szóköz. A validációs üzenetek beépítettek, de lehetőség van a változtatásukra. Személy szerint ebben a keretrendszerben használható fentebb említett modul miatt ezt a megoldást találtam a legkényelmesebbnek a rendelkezésre álló szabályok sokszerűsége miatt.\\
\subsubsection{React.js:} A form validáció implementálása ebben a keretrendszerben volt a legnehezebb és a legtöbb időt ez vette igénybe, lévén, hogy nincs kétirányú adatkötés, így különböző validációs funkciókat kellett létrehozni a kívánt eredmény eléréséhez, továbbá ennél a megoldásnál a validációs hibaüzenetek csak a kitöltendő form felett lévő panel-ban jelennek meg.\\