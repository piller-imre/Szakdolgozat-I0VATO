\Chapter{Bevezetés}

Napjainkban nagyon sokféle megoldás létezik egy weboldal elkészítésére. Keretrendszerek és könyvtárak egyre növekvő tárházából választhatjuk ki a számunkra legmegfelelőbb megoldást.\\ Fontos, hogy jól dokumentált és elterjedt legyen, továbbá kiszolgálja a fejlesztő vagy cég igényeit. A szakdolgozatomban a négy legnépszerűbb és legelterjedtebb keretrendszert hasonlítom össze, melyek a következők: AngularJS, Angular 2, Vue.js és React.\\ A dolgozat első részében az MVC keretrendszer, az ECMAScript és a JavaScript technológiák bemutatása kerül előtérbe, ezt követően bemutatásra kerül az elkészített alkalmazás specifikációja, majd az alkalmazások külön-külön, mintakódokkal való szemléltetése történik meg.\\ Minden keretrendszernél egy projekt struktúra hozzáadása történik meg, amely szemlélteti az adott alkalmazás mappaszerkezetének felépítését.
Az alkalmazások fő témája az MMA Fighters (harcosok) adatait kezelő webes felületek részletes bemutatása. A mintaalkalmazások kifejtésénél a különböző funkciók aktuális technológiabeli megoldásait mutatom be.\\Ezek a funkciók a CRUD műveletek megvalósítása, template-k, az oldalak közötti routing működése, szűrők és direktívák használata, form validáció, valamint az Auth0 szerviz használatával bejelentkezési és regisztrációs lehetőségek megteremtése a felhasználók számára.\\ A dolgozat záró fejezetében a keretrendszerek összehasonlítása után az olvasó képet kap az egyes keretrendszerek előnyeiről, illetve hátrányairól.