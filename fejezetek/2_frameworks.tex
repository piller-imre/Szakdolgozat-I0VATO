\Chapter{JavaScript technológiák és keretrendszerek}

\Section{MVC keretrendszer}

Az MVC keretrendszer (Model-View-Controller) egy olyan felépítési minta, amelynek segítségével az alkalmazásokban lévő adatokat több részre bontjuk, egészen pontosan a modellre, a nézetre és a vezérlőre. A fő cél az alkalmazás rétegeinek egymástól való elkülönítése. Lényege, hogy az egyes részekben lévő adatok változtatását egyszerűbbé teszi a különválasztás által, mert így nem zavarják egymást a komponensek. \\A modellben tároljuk azokat az adatokat, amin a vezérlő kódja műveleteket végez és amelyet a nézet által megjelenít. Ilyenek lehetnek például a form-ok (űrlap) gombok, linkek, navigációs sávok, táblázatok, listák. A felhasználó a "Delete" gombra kattintás esetében azt látja, hogy az oldalon megjelenített adatokból eltűnt az, amit éppen kitörölt. Ezt a vezérlőben lehet meghatározni, lekezelni az egyes eseteket, azt hogy a felhasználó interakciója a felülettel mikor milyen eseménnyel, változtatással járjon. A modell-t általában fájlokban, vagy adatbázisban tárolják. Ez a rész tartalmazza az adatok logikai felépítését, de a felhasználó felületről semmilyen fajta információt nem tárol. A vezérlő áll kapcsolatban mind a modell-el, mind a nézettel.

CRUD művelete(CREATE, READ, UPDATE, DELETE) – Létrehozás, Lekérés, Frissítés, Törlés.
Az elkészített webalkalmazásban ezeket a funkciókat az MMA harcosok adatain lehet végrehajtani.


\Section{Az ECMAScript szabvány}

A JavaScript-et Brendan Eich találta fel 1995-ben, és 1997-ben lett ECMA szabvány.
A szabvány hivatalos neve ECMA-262, az ECMAScript pedig a hivatalos neve a nyelvnek.

\begin{tabular}{|l|l|p{8cm}|}
\hline
\textbf{Év} & \textbf{Név} & \textbf{Leírás} \\
\hline
1997 & ECMAScript 1 & Első kiadás \\
\hline
1998 & ECMAScript 2 & Csak szerkesztőségi változtatások \\
\hline
1999 & ECMAScript 3 & Hagyományos kifejezések és try/catch (hibakezelés) hozzáadva \\
\hline
- & ECMAScript 4 & Sosem jelent meg \\
\hline
2009 & ECMAScript 5 & ,,Szigorú mód'' és JSON támogatás hozzáadva \\
\hline
2011 & ECMAScript 5.1 & Szerkesztőségi változtatások \\
\hline
2015 & ECMAScript 6 & Osztályok és modulok hozzáadva \\
\hline
2016 & ECMAScript 7 & Exponenciális operátor (**) és Array.prototype.includes hozzáadva \\
\hline
2017 & ECMAScript 8 & await/async funkciók, amelyek generátorok, ígéretek (promise) használatával működnek \\
\hline
\end{tabular}
\\
\cite{ECMAScript}

A JavaScript a Netscape nevű cég fejlesztése, az első böngésző, ami futtatni tudta a JavaScript-et a Netscape 2 volt 1996-ban. A Netscape után a Mozilla cége folytatta a fejlesztését a Firefox nevű böngészőjének. A JavaScript verziószámok 1.0-tól vannak számozva, az utolsó verzió száma: 1.8.5

Az ECMAScript egy nyelvi standard, az ECMA International cég fejlesztése, a JavaScript implementációból fejlődött ki. Az első kiadása 1997-ben jelent meg, a verziószámai 1-től 7-ig vannak sorszámozva.
A JScript-et a Microsoft nevű cég fejlesztette ki 1996-ban, mint egy kompatibilis JavaScript nyelvet az Internet Explorer böngészőjükhöz. A JScript verziószámai 1.0-tól 9.0-ig terjednek.

\Section{JavaScript technológiák áttekintése}

Az alábbi fejezetben a legelterjedtebb JavaScript technológiákról lesz szó. Olyanokról, mint: AngularJS, Angular 2, Vue.js, React. Mintaalkalmazásomban ez a négy technológia került implementálásra.

\SubSection{AngularJS} 

Előnyei:

\begin{itemize}
\item Template-ek használata, 
\item a kétirányú adatkötés segítségével szinkronizálja a DOM-ot (Document Object Model) és a modell-t
\item belső HTML kódok hozzáadása helyett, azonnal a DOM-ot módosítja az oldalon, így gyorsabb, 
\item MVC / MVVN (Model-View-ViewModel) használata, de közelebb áll az MVVN-hez,
\item DI (Dependency Injection) - alkalmazásbeli függőségek kezelése,
\item Google cég által fejlesztett és támogatott, hatalmas fejlesztői táborral rendelkezik,
\item saját direktívák létrehozásának lehetősége,
\item a tesztelésre fordított figyelem, a tesztelés egyszerűsége.
\end{itemize}

Hátrányai: 

\begin{itemize}
\item Az Angular 2-t teljesen újraírták így visszafelé nem kompatibilis, azok a fejlesztők, akik Angular 2-t akarnak használni, az alapoktól kell újraírniuk az alkalmazásukat,
\item nem támogatja az oldal betöltésének felgyorsítását (szerver-oldali renderelés),
\item a JavaScript támogatás kikapcsolása esetén nem lehet elérni a weboldalt,
\item a scope-k, és a direktívák használata a kezdők számára bonyolult lehet. \cite{AngularJS}
\end{itemize}

\SubSection{Angular 2} 

Előnyei:

\begin{itemize}
\item Angular 2 CLI-vel (Command Line Interface) az alapból működő alkalmazások létrehozása egyszerűvé válik, 
\item komponens alapú használat,
\item a TypeScript a JavaScript továbbfejlesztése,
\item több platformon elérhető, Ionic keretrendszer használatával hibrid mobilalkalmazások létrehozását teszi lehetővé,
\item elterjedt kombináció (MEA2N)- MongoDB-Express.js-Angular2-Node.js  
\item jobb teljesítmény és gyorsaság az AngularJS-hez képest,
\item animimációs csomag használatának lehetősége, így a fejlesztőnek nem kell animációs könyvtárakat megtanulnia.
\end{itemize}

Hátrányai: 
\begin{itemize}
\item A nyelvnek a megtanulása (TypeScript) több időt igényel azoknak, akik előtte még nem ismerték, 
\item a DOM közvetlenül történő változtatása miatt lassabb, 
\item kevés TypeScript és Angular 2 fejlesztő van, aki igazán jó fejlesztő lenne. \cite{Angular 2}
\end{itemize}


\SubSection{Vue.js} 

Előnyei:

\begin{itemize}
\item A keretrendszer használatának megértése és a benne történő fejlesztés egyszerű,
\item mérete nagyon kicsi, 30 KB összesen a .gzip úgynevezett "minified", vagyis tömörített változat
\item opcionális JSX támogatás,
\item kétirányú adatkötés "v-model" használatával,
\item gyors fejlesztés,
\item Vue CLI (Command Line Interface) használata.
\end{itemize}

Hátrányai:

\begin{itemize}
\item A keretrendszer használatának megértése és a benne történő fejlesztés egyszerű,
\item mérete nagyon kicsi, 30 KB összesen a .gzip úgynevezett "minified", vagyis tömörített változat
\item opcionális JSX támogatás,
\item kétirányú adatkötés "v-model" használatával,
\item gyors fejlesztés,
\item Vue CLI (Command Line Interface) használata. \cite{Vue.js}
\end{itemize}

\SubSection{React} 

Előnyei:

\begin{itemize}
\item Komplexebb web-es alkalmazások készítésére is alkalmas,
\item Create-React-App CLI az alkalmazások gyors fejlesztéséhez,
\item elterjedt (MERN) - (MongoDB-Express.js-React-Node.js),
\item csak akkor frissíti a virtuális DOM-ot, ha szükséges, emiatt nagyon gyors,
\item szerver és kliens oldali renderelés,
\item komponens alapú, a komponenseket beágyazhatjuk, újra felhasználhatjuk,
\item React Native technológiát használ, amellyel teljes értékű mobilalkalmazásokat lehet készíteni.
\end{itemize}


Hátrányai:

\begin{itemize}
\item Hiányzik belőle a kétirányú adatkötés, így az adatváltoztatások esetében  saját kód írásával kell orvosolni a problémát,
\item csak a "Nézet" (M-View-C) részt képviseli, így többféle könyvtár hozzáadását igényli a különböző feladatok megoldásához,
\item tanulása a JSX és ECMAScript 6 miatt több energiát vesz igénybe,
\item teljesen más gondolkodást igényel az MVC mintáktól,
\item rendesen kidolgozott hatékony kombinációkat találni a React-al történő fejlesztéshez nem egyszerű. \cite{React}
\end{itemize}



\SubSection{TypeScript}

\begin{itemize}
\item Egy ingyenes és nyílt forráskódú programozási nyelv,
\item a JavaScript továbbfejlesztése, 
\item arra lett kitalálva, hogy az objektum-orientált programozási nyelvekben jártas fejlesztőknek ne kelljen a JavaScript-hez szükséges gondolkodásmódra átállniuk, 
\item JavaScript kódra fordít, amely bármely JavaScript motorban és böngészőben képes futni, amelyik támogatja az ES3-at (ECMAScript 3).
\end{itemize}


\SubSection{CoffeeScript}

\begin{itemize}
\item Egy nyelv, ami JavaScript kódra fordít,
\item javítani próbál a JavaScript-en, kevesebb kóddal ugyanazt az eredményt elérni,
\item könnyen olvasható, érhető és gyors.
\end{itemize}

\Section{JavaScript implementációk}

ECMAScript motorok: olyan programok, amik végrehajtanak olyan forráskódokat, amik az ECMAScript nyelvi standardjaiban, például JavaScript-ben íródtak.

Carakan: Egy JavaScript motor, amit az Opera Software ASA fejleszt, a 10.50-es verziószámú Opera webböngészőben volt, amíg nem váltottak a V8-ra az Opera 15 verziójával, ami 2013-ban jelent meg.

Chakra (JScript9): A JScript motor az Internet Explorer-ben volt használatos. Először a MIX 10-ben lett bemutatva.

Chakra: JavaScript motor, amit a Microsoft Edge-ben használnak.

SpiderMonkey: Egy JavaScript motor a Mozilla Gecko alkalmazásaiban, beleértve a Firefox-ot. Tartalmazza az IonMonkey fordítót és az OdinMonkey optimalizációs modult.

JavaScriptCore: Egy JavaScript értelmező és JIT (just-in-time compilation), a KJS fejlesztése. A WebKit projekt kereteiben belül használatos olyan alkalmazásokban, mint a Safari.

Tamarin: Egy ActionScript és ECMAScript motor, amit az Adobe Flash használ.

V8: JavaScript motor, amit a Google Chrome, a Node.js és a V8.NET használ.

Nashorn: JavaScript motor, az Oracle Java Development Kit (JDK) használja
a 8-as verziótól. \cite{ECMAScript motorok}
