\Chapter{Auth0 szervíz}

Az Auth0 egy olyan platform, amit a fejlesztők használhatnak arra, hogy alkalmazásaikba autentikációt és autorizációt ültessenek. Támogatja a legelterjedtebb keretrendszereket és technológiákat, dokumentációja átlátható. Továbbá egy folyamatosan fejlődő, innovatív rendszer. \cite{Auth0}

Az egyes keretrendszerek programjaiban való alkalmazása a következő módon történik: (home.component.html fájl)

\subsubsection{Angular 2 esetében:}

\begin{cpp}
<h4 *ngIf="auth.isAuthenticated()">
    <div>
   	<a [routerLink]="['/fighter-create']">ADD NEW FIGHTER</a>
   	</div>
</h4>
<h4 *ngIf="!auth.isAuthenticated()">
  You are not logged in! Please 
  <a (click)="auth.login()">Log In</a> to continue.
</h4>
\end{cpp}

Az fenti kód az "*ngIf" direktíva használatával meghívja az "auth" "isAuthenticated()" függvényét, amely leellenőrzi, hogy a felhasználó be van e jelentkezve és ha igen, akkor megjeleníti az "ADD NEW FIGHTER" nevű linket. Ha a felhasználó nincs bejelentkezve, akkor a "You are not logged in" üzenet és a "Please Log In to continue" üzenet jelenik meg, ahol a "Log In" egy gomb, ami meghívja az "auth" "login()" függvényét.

Az Auth0 szerviz használatához szükség van egy felhasználói fiók létrehozására az Auth0 hivatalos weboldalán, majd a megfelelő keretrendszer kiválasztása után az adott keretrendszerre nézve szükséges végrehajtani a leírt lépéseket, hogy az alkalmazásunkba integrálhassuk a szervizt.

A "home.component.ts" fájlban szükséges importálni a következő parancsot:
\begin{cpp}
import { AuthService } from './../auth/auth.service';
\end{cpp}

Továbbá a konstruktorban definiálni az "auth"-ot, mint az előbb beimportált\\ "AuthService" nevű szervizt:
\begin{cpp}
export class HomeComponent implements OnInit {

  constructor(public auth: AuthService) { }
  ngOnInit() {
  }
}
\end{cpp}

A felhasználó azonosításától függően megjeleníteni kívánt tartalom kódja a többi keretrendszer esetében is hasonló elven működik. Használata direktívákkal történik. Minden keretrendszerhez megtalálható a megfelelő dokumentáció az Auth0 hivatalos oldalán. \cite{Auth0}

% TODO: Ez egy külön fejezetnek nagyon rövid így. Az authentikáció úgy általában fontos, tehát jó, ha külön fejezet, viszont akkor a többi keretrendszerről is érdemes valamit írni ennek kapcsán. (Vagy azt is lehet, hogy minden keretrendszer végére kerül még be néhány észrevétel ezzel kapcsolatban.)
