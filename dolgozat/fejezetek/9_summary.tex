\Chapter{Összegzés}

A dolgozatomban a jelenleg legnépszerűbb és legelterjedtebb JavaScript keretrendszereket hasonlítom össze. A dolgozat elején az MVC keretrendszert, a JavaScript és ECMAScript közötti különbséget, a verziókat és történelmi áttekintőt mutatok be. Ezután az a JavaScript keretrendszerek néhány tulajdonságát írtam le, mint előnyök és hátrányok.

A dolgozat fő témája a keretrendszerek több szempontból való összehasonlítása mintakódok alapján. Ezek a szempontok a CRUD műveletek (Create, Read, Update, Delete) megvalósítása, template-k és a weboldalak közötti routing működése, szűrők és direktívák használata, form validáció, valamint bejelentkezés és regisztráció. 
A négy webalkalmazás AngularJS, Angular 2, Vue.js, és React.js nyelveken készült el. A weboldalak az MMA harcosokkal foglalkoznak (MMA Fighters).

Véleményem az AngularJS keretrendszer volt a legkönnyebben megérthető, a tesztelése a különböző részeknek egyszerűen megoldható és a kétirányú adatkötés használata megkönnyíti a fejlesztést, de az Angular 2-nél és a React.js-nél a komponensek generálásának lehetősége jelentősen lecsökkentette a fejlesztési időt.
