%Az összefoglaló fejezet
\chapter*{Adathordozó használati útmutató}
\addcontentsline{toc}{chapter}{Adathordozó használati útmutató}

A szakdolgozatomban elkészítettem négy webalkalmazás programot, amelyek a "JavaScript keretrendszerek" mappában találhatók. Az elkészített alkalmazások az alábbi keretrendszerekben kerültek implementálásra: AngularJS, Angular 2, React.js, Vue.js, így az almappák nevei ezekkel egyeznek meg. Az alkalmazások az MMA harcosokkal foglalkoznak, ezért minden keretrendszer mappájában az "mma\_in\_keretrendszer neve" \\ konvenciót használtam.\\

A keretrendszereket külön-külön részletezem. \\

Az AngularJS esetében az "mma\_in\_angularjs" nevű mappában találhatók a fájlok. A szerver indításához szükséges a Node.js program letöltése, telepítése. A harcosok adatbázisban való eltárolásához pedig a MongoDB nevű program letöltésére és telepítésére van szükség annak érdekében, hogy a létrehozott harcosok az oldal lefrissítése után is megmaradjanak. \\(Ha a MongoDB telepítésekor nincs megadva, hogy a szerver automatikusan elinduljon, akkor a programok elindítása előtt szükséges a "mongod" nevű parancs kiadása, amelyet a MongoDB telepítési helyén lévő "bin" mappába való navigálás után lehet megtenni.) A "Node.js command prompt" nevű parancssoros alkalmazás megnyitása után be kell lépni a "JavaScript keretrendszerek" nevű mappába, majd az "mma\_in\_angularjs" nevű mappába. Ezt a "cd" paranccsal tehetjük meg, például "cd JavaScript keretrendszerek". A meghajtóváltás parancsa a "D:", vagy "C:", ez függ a meghajtó betűjelétől. \\Minden további keretrendszernél is ugyanezeket a műveleteket kell elvégezni. \\Ha az "mma\_in\_angularjs" nevű mappába sikerült belépni, ki kell adni az "npm install" parancsot, amely feltelepíti a szükséges modulokat, majd az "npm start" parancsot, ami elindítja a szervert. A "http://localhost:3000" URL című oldalon megjelenik az "MMA Fighters" weboldal, ez esetben az AngularJS-ben írt verzió.\\
A főoldalon, a navigációs sávon lévő "Login" gombbal lehet bejelentkezni, itt a Facebook vagy Google szolgáltatások közül lehet választani, vagy a "Sign Up" gombra kattintva, az email cím és jelszó beírásával új felhasználót lehet létrehozni. A "Log In" gombra kattintva a program betölti a főoldalt, ahol két gomb található: az "ADD NEW FIGHTER" és a "VIEW THE FIGHTERS". Az "ADD NEW FIGHTER" nevű gombra kattintva megjelenik az új harcost létrehozó oldal, ahol a mezők kitöltése után a "Submit" gombra kattintva létre lehet hozni egy új harcost, az oldal ezt követően a harcosokat megjelenítő táblázathoz irányít át. A "Go Back" nevű gombra kattintás után szintén a harcosok listája jelenik meg. Az új harcos létrehozására szolgáló oldal a navigációs sávon lévő "Add new fighter" nevű gombbal is behozható. \\A "VIEW THE FIGHTERS" nevű gombra kattintva pedig szintén a harcosokat megjelenítő táblázat jelenik meg. Itt a táblázatban minden harcos sorának végén az "Action", vagyis "Esemény" felirat alatt lévő "View Details" linkre történő kattintás után az adott harcosnak az adatait megjelenítő oldal jön be. \\Itt az "Edit" gombra kattintva a harcos adatait módosító oldal jelenik meg, ahol a mezők értékeinek változtatása után a "Submit" gomb lenyomására a harcosokat megjelenítő lista oldala jelenik meg, ahogy a "Delete" gomb és a "Go Back" gomb megnyomása után is. A "Go to fighter page" gombra való kattintás a harcos "page\_url" mezőjében megadott weboldalra navigál.
\\Végül a navigációs sávon lévő "Logout" gombra kattintva a kijelentkezés utáni kezdőoldal jelenik meg, ahol a "Login" gomb található.
Ez a felépítés mind a négy keretrendszer esetében az itt leírtaknak felel meg, viszont a projekt elindítása változik a mappába való navigálást követően.
\\

Az Angular 2 esetében az "Angular 2" mappába és az "mma\_in\_angular2" mappába való belépés után ki kell adni az "npm install" parancsot, amely feltelepíti a szükséges modulokat, majd az "npm start" parancsot, ami elindítja a szervert. Ezután egy újabb "Node.js command prompt" nevű parancssoros alkalmazást kell nyitni. Az "npm install" parancs utáni esetleges hibák elkerülése végett célszerű adminisztrátorként indítani a programot. A program ezen ablakában a "cd src" parancs, majd az "ng build" parancs kiadása szükséges. Ezután a "http://localhost:3000" URL című oldalra lépve jelenik meg az Angular 2 keretrendszerbeli alkalmazás. \\

A Vue.js esetében az "mma\_in\_vuejs" nevű mappába való navigálás után a "server" nevű mappába történő belépés szükséges, ahol ki kell adni az "npm install", majd az "npm start" parancsokat, mint az előző keretrendszerek esetében. Ugyanezeket a parancsokat szintén ki kell adni az "mma\_in\_vuejs" nevű mappában lévő "client" almappába való belépés után is. Ezután a "http://localhost:8080" URL című oldalra navigálva jelenik meg a Vue.js-ben írt alkalmazás.\\

A React.js esetében az "mma\_in\_reactjs" mappába történő belépés után az "npm install" parancsot kell beírni, majd az "npm run build", majd az "npm start" parancsok kiadását követően a "http://localhost:3000" URL című oldalra navigálva megjelenik a React.js-ben írt alkalmazás.


